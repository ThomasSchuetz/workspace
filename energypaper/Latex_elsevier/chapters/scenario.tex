\section{Case study}
\label{sec:Case study}

This section describes a case study in which the effects of the implemented German regulations on the optimal BES configuration as well as resulting annualized costs and emissions are analyzed.

The following subsections summarize the used inputs and the obtained results.
Further information on both as well as the implemented model can be found in this project's open-source repository\footnote{\url{https://github.com/RWTH-EBC/BESopt}}.

\subsection{Inputs}

The inputs consist of the building's energy load profiles, such as electricity demands of plug loads, domestic hot water usage and space heating loads.
Additionally, the considered energy conversion units are described in this subsection.

\subsubsection{Energy load profiles}

We applied the optimization model described in the previous section to a residential building located in Aachen (Germany) with 4 residents and a total of 255~m$^2$ heated floor area.
The hourly space heating profiles are computed with the model verified in \textbf{CITE APPLIED ENERGY PAPER}.
Electricity demands for non-heating devices, appliances and lighting are computed with a high-resolution, stochastic tool based on Richardson et al. \cite{Richardson2010}.
Domestic hot water demand profiles are computed with a combination of users' occupancy based on Richardson et al. \cite{Richardson2010} and daily tap water usage statistics of residential buildings developed in IEA Annex 42 \cite{IEAEnergyConservationinBuildingsCommunitySystems}.
The cumulated, annual space heating and domestic hot water demands result in 23500~kWh and the annual electricity demand is 4000~kWh.

\subsubsection{Energy conversion units}

All energy conversion units that have been modeled are based on manufacturers' data sheets and price recommendations from retailers or original manufacturers.
We consider between 7 and 12 different types of electrical and gas-fired heating devices.
Their ranges regarding nominal heat output and investment costs are shown in Table~\ref{tab: parameters heating devices}.
Similarly, Table~\ref{tab: parameters solar devices} summarizes the solar devices and inverters, and Table~\ref{tab: parameters storage devices} lists the ranges of storage devices.


\begin{table}[h!]
	\caption{Summary of considered heating devices}
	\centering
	\begin{tabular}[l]{@{}lccc}
		\hline
		Component 	& No. types	& $\dot{Q}^\mathrm{nom}$& $c^\mathrm{inv}$ \\
					&  			& kW 					& 1000 Euro \\
		\hline
		CHP 		& 7 		& 0.70 -- 45.00 		& 8.93 -- 62.07\\
		BOI 		& 8 		& 14.20 -- 66.30 		& 1.33 -- 4.26\\
		EH 			& 12 		& 2.00 -- 12.00 		& 0.17 -- 0.22\\
		HP 			& 8 		& 3.25 -- 50.00 		& 3.94 -- 25.44\\
		\hline
	\end{tabular}
	\label{tab: parameters heating devices}
\end{table}

\begin{table}[h!]
	\caption{Summary of solar devices and inverters}
	\centering
	\begin{tabular}[l]{@{}lccc}
		\hline
		Component 	& No. types	& $P^\mathrm{nom}$/$\dot{Q}^\mathrm{nom}$& $c^\mathrm{inv}$ \\
		&  			& kW 					& 1000 Euro \\
		\hline
		PV 			& 9 		& 0.14 -- 0.23 		& 0.16 -- 0.26\\
		STC 		& 10 		& 0.42 -- 1.32 		& 0.19 -- 0.82\\
		INV			& 27 		& 1.60 -- 28.60 	& 0.59 -- 3.67\\
		\hline
	\end{tabular}
	\label{tab: parameters solar devices}
\end{table}

\begin{table}[h!]
	\caption{Summary of storage devices}
	\centering
	\begin{tabular}[l]{@{}lccc}
		\hline
		Component 	& No. types	& Capacity 		& $c^\mathrm{inv}$ \\
					&  			& kWh / m$^3$ 	& 1000 Euro \\
		\hline
		BAT 		& 9 		& 3.60 -- 11.60		& 6.00 -- 14.33\\
		TES 		& 14 		& 0.12 -- 2.00 		& 0.42 -- 1.71\\
		\hline
	\end{tabular}
	\label{tab: parameters storage devices}
\end{table}

\subsubsection{Tariffs and emissions}

The electricity and gas tariffs' characteristics are listed in Table~\ref{tab: parameters tariffs}.
All tariffs are derived from German utility providers.
As shown in Table~\ref{tab: parameters tariffs}, standard tariffs are slightly less expensive than eco tariffs, but cause significantly more CO$_2$ emissions.
The implemented heat pump tariff further offers a strong economic incentive compared to the other tariffs.


\begin{table}[h!]
	\caption{Summary of storage devices}
	\centering
	\begin{tabular}[l]{@{}lccc}
		\hline
		Tariff 	& $c^\mathrm{fix}$	& $c^\mathrm{var}$ 		& $emi^\mathrm{spec}$ \\
		& Euro / a 	& Euro / kWh & kg CO$_2$ / kWh \\
		\hline
		Gas (standard) & \multicolumn{1}{r}{39.98 -- 182.50} & \multicolumn{1}{r}{0.080 -- 0.058} & 0.250 \\
		Gas (eco) & \multicolumn{1}{r}{39.98 -- 182.50} & \multicolumn{1}{r}{0.083 -- 0.061} & 0.238 \\
		\hline
		El. (standard) & \multicolumn{1}{r}{73.02 -- 84.16} & 0.266 -- 0.263 & 0.569 \\
		El. (eco) & \multicolumn{1}{r}{73.02 -- 117.73} & \multicolumn{1}{r}{0.275 -- 0.265} & 0.329 \\
		El. (heat pump) & \multicolumn{1}{r}{91.51}  & \multicolumn{1}{r}{0.196} & 0.508 \\
		\hline
	\end{tabular}
	\label{tab: parameters tariffs}
\end{table}

\subsection{Scenarios}

We analyze how the implemented regulations affect the optimal BES configuration, therefore we consider for each regulation one scenario in which the respective regulation is activated and a comparison in which it is deactivated.

In the \emph{PV with EEG scenario}, the solver has to install at least one PV module and simultaneously the restrictions of the Renewable Energy Sources Act, such as the fixed feed-in remuneration and limited feed-in, are enforced.
In the \emph{PV without EEG scenario}, the solver also has to install at least one PV module but neither feed-in subsidies nor limits on the PV feed-in are set.
For both scenarios, no additional constraints on the heat generation system are used, allowing the optimizer to choose any available devices.

The \emph{CHP with KWKG scenario} requires the installation of a CHP unit while accepting the subsidies according to the Cogeneration Act for feed-in and self-consumed electricity.
On the other hand, in the \emph{CHP without KWKG scenario}, the solver is forced to install a CHP unit without receiving any subsidies for electricity generated with the CHP unit.

The \emph{HP with tariff scenario} enforces the installation of a HP and choosing a special HP tariff.
In the \emph{HP without tariff scenario}, a HP is installed, but a regular electricity tariff is chosen.

Finally, the \emph{BAT with KfW scenario} analyzes the installation of a BAT with the KfW support program that provides financial incentives to lower feed-in limits of electricity generated with PV.
In the \emph{BAT without KfW scenario}, a BAT is installed, waiving the financial support through KfW but keeping the right to export electricity from PV according to the limits based on the Renewable Energy Sources Act.

Additionally, we use two different methods to compare the results in a broader context.
The first reference is the \emph{traditional scenario} forcing a traditional BES design that only allows for installing a boiler to cover the heat demand and purchasing the entire electricity demand from the grid.
While all previously mentioned scenarios minimize the total annualized costs, the second reference is the \emph{Pareto-Set} of the multi-objective optimization minimizing annualized costs and emissions.
This multi objective optimization is based on the $\varepsilon$-constraint method described by Mavrotas \cite{Mavrotas2009}.
The \emph{traditional scenario} is still a common design paradigm and the \emph{Pareto-Set} serves for assessing the economical and ecological results of the other scenarios on a global optimal scale.

These ten different scenarios are summarized in Table~\ref{tab: scenarios setup}.

\begin{table}[h!]
	\caption{Summary of investigated scenarios}
	\centering
	\begin{tabular}[l]{@{}lll}
		\hline
		Scenario 	& Device selection	& Subsidies / restrictions \\
		\hline
		PV with EEG & At least one PV module & Fixed above market feed-in remuneration\\
		& & Max. 70\% of PV peak power can be fed into the grid\\
		PV without EEG & At least one PV module & Market based feed-in remuneration\\
		& & No limitations on feed-in\\
		\hline
		CHP with KWKG & Force CHP purchase & Subsidies for feed-in and self-consumption from CHP\\
		CHP without KWKG & Force CHP purchase & No subsidies for electricity from CHP units\\
		\hline
		HP with tariff & Force HP purchase & Select reduced HP tariff\\
		HP without tariff & Force HP purchase & Prohibit reduced HP tariff\\
		\hline
		BAT with KfW & Select a BAT & Financial support according to KfW\\
		& & Max. 50\% of PV peak power can be fed into the grid\\
		BAT without KfW & Select a BAT & No financial support for BAT\\
		& & Max. 70\% of PV peak power can be fed into the grid\\
		\hline
		Traditional & Only BOI and TES & EEG, KWKG, KfW and free choice of electricity tariff \\
		Pareto-Set & No presetting &  EEG, KWKG, KfW and free choice of electricity tariff \\
		
	\end{tabular}
	\label{tab: scenarios setup}
\end{table}

\subsection{Computing hardware}

All computations are carried out with Gurobi~6.5\footnote{\url{http://www.gurobi.com/index}} and modeled with the corresponding Python framework.
We used the Branch-and-Cut \cite{Wolsey1998} algorithms from this solver.
The optimizations are solved to an optimality gap of 1\% on a Windows~7 computer with 6~CPU cores, 12~threads and 32~GB of RAM.

