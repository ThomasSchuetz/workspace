\section{Introduction}

The transition towards a more energy efficient and environmentally friendly economy is a recognized objective of the European Union \cite{TheEuropeanParliamentandtheCouncil2010}.
In Germany, this concept is known as ``Energiewende'' and aims at reducing greenhouse gas emissions, increasing electricity generation from renewable energy sources (RES) and achieving higher energy efficiency in general \cite{FederalMinistryforEconomicAffairsandEnergy2012}. 
In the context of buildings, which account for approximately 40\% of total energy consumption in the European Union \cite{TheEuropeanParliamentandtheCouncil2010}, emission reductions and energy savings can for example be achieved by installing more efficient heating devices and by improving their control strategy.

In recent years, many different heat and electricity generation as well as storage technologies evolved for application in buildings.
Small scale Combined Heat and Power (CHP) units offer a highly efficient method for generating heat and electricity simultaneously from fossil fuels \cite{Maghanki2013,Peacock2005}.
Potential benefits can further be leveraged by introduction of Thermal Energy Storage (TES) devices \cite{Haeseldonckx2007}.
Heat Pump (HP) systems present a technology for efficiently using electricity for heating purposes \cite{Staffell2012}.
Renewable Energy Sources (RES), especially solar systems can also be used on building level, such as for example Solar Thermal Collectors (STCs) \cite{Kalogirou2004} or Photovoltaic (PV) modules \cite{Parida2011}.
The integration of such fluctuating generators can for example be enhanced by storage devices such as TES \cite{Tian2013} and batteries \cite{Luthander2015}.

In order to achieve the proposed emission reductions and energy savings, the German government supports the utilization of environmentally friendly technologies such as RES and CHP in the building sector.
For example, the Renewable Energy Sources Act \cite{EEG2014}, guarantees above market and long-term feed-in tariffs for PV plants.
Additionally, the Combined Heat and Power Act \cite{KWKG2016} provides subsidies for feed-in and self-consumed electricity from CHP.
Further methods for promoting RES and eco-friendly technologies include subsidies from the German Reconstruction Credit Institute for combined PV and battery systems \cite{KfW275_2016} as well as private utility companies offering reduced electricity tariffs for heat pump (HP) systems.

The vast quantity of combinations of available devices into a Building Energy System (BES) requires a systematic analysis and evaluation method \cite{Yang2015}, making optimization approaches a viable option for determining the optimal structure, sizing and operation of BES.
The inherent nonlinearities arising in such optimization models, like the typically nonlinear part load behavior of generation units, led to the development of mixed-integer nonlinear programs (MINLP) \cite{Pruitt2013,Ren2008}.
However, the majority of analyses dealing with the optimal design and operation of BES consider mixed-integer linear programs (MILP), often requiring large simplifications.
These simplifications most frequently occur regarding the devices' capacity and their part load behavior.
For example, Ashouri et al. \cite{Ashouri2013} presented a MILP framework for the optimal selection and sizing of smart building systems considering all aforementioned generation and storage units, as well as chillers and ice storage systems. 
Their device modeling considers continuous equipment sizes rather than available, discrete sizes.
Furthermore, no switching on threshold or decreased part load performance are modeled.
Other studies \cite{Ameri2016,Voll2013} use the continuous dimensioning as done by Ashouri et al. \cite{Ashouri2013}, however accounting for linear part load losses.
Ameri and Besharati \cite{Ameri2016} optimize district heating and cooling networks considering gas turbines, boilers, chillers and PV.
Voll et al. \cite{Voll2013} compute an optimal energy system for industrial applications comprising CHPs, boilers and chillers.
Mehleri et al. \cite{Mehleri2012,Mehleri2013} model some equipment sizes, such as boilers, PV area and TES volume with continuous variables, whereas the CHP capacity is described with discrete steps.
They also neither account for switching on thresholds nor part load deterioration.
Their model is applied to optimize local neighborhoods considering CHPs, boilers, PV, TES and district heating as well as microgrids.
Such a mixed modeling of device capacities is further used by \cite{Harb2016,Lozano2010}.
Harb et al. \cite{Harb2016} model CHP and HP capacities discretely and rely on the continuous sizing for PV, boilers and TES.
In this model, CHP part load is handled with an empirical model, whereas constant efficiencies are assumed for HP and boilers for the entire modulation range.
The model is applied to determine optimal configurations for German residential buildings and extended to compute local heating networks and microgrids for a small neighborhood.
To the best of our knowledge, this approach presents the only available model for considering multiple electricity tariffs during the design optimization, in particular a special heat pump tariff and a standard electricity tariff.
Lozano et al. \cite{Lozano2010} present the optimization of combined heat, cooling and power systems considering CHP, boilers, chillers and TES.
They model device capacities discretely but TES sizes continuously.
Their devices are assumed to be 2-point controlled, not modeling part load.
Buoro et al. \cite{Buoro2012} analyze standard and domotic homes by optimizing their respective energy system consisting of CHP, boiler, chiller, PV, STC and TES.
The part load is based on a linear regression without accounting for the devices' activation threshold.
Renaldi et al. \cite{Renaldi2016} describe a framework for optimizing HP and TES systems for residential buildings.
Their devices' selection is entirely based on discrete choices, however their TES model assumes permanent losses for each device, even if the storage is totally discharged.
Heat pumps' temperature dependent COP is modeled, however the COP's deterioration during part load is neglected.
Wakui and Yokoyama \cite{Wakui2014} present a model for optimizing energy systems for residential buildings comprising CHP, TES, boiler and electrical water heaters.
The selection of CHP and storage tanks is coupled, so that if a certain CHP unit is chosen, a pre-specified storage tank is also installed as well.
Part load behavior is described by means of piecewise linearization, introducing multiple linear relationships that model the nonlinear part load curves.
This model has been extended in \cite{Wakui2016} by further considering HP units.
The selection of TES units has been decoupled from the selection of CHP units in this work; however, TES' sizes are chosen as a continuous variables rather than discretely.

In order to contribute to the field of optimal design of building energy systems, we present a MILP framework for determining the design, size and operation of building energy systems comprising CHP, HP, boiler, electrical heaters, TES, batteries, PV and STC.
Based on the previously described works, our paper presents three novelties.
First, we model the selection of all devices, including TES, discretely.
In this way, we assure that the determined, optimal system is available for purchase in reality, and we are able to include detailed information such as efficiency curves and investment costs for each type of device.
As illustrated previously by Wakui et al. \cite{Wakui2014,Wakui2016} for CHP units, we use multiple piecewise linearizations for all considered CHPs, HPs and boilers in this work.
The second novelty is a detailed modeling of many specific German regulations and market characteristics.
We account for limitations on PV feed-in based on the current Renewable Energy Act from 2014, we model the German Cogeneration Act from 2016 and consider subsidies for PV systems with batteries as well as reduced heat pump tariffs.
Our third contribution is the design decision between multiple gas and electricity tariffs.
Previous studies only accounted for a single gas and electricity rate, whereas we include the possibility of choosing more expensive eco-tariffs to analyze the trade-off between monetary costs and environmental impacts.

These novelties are of manifold importance.
Researchers benefit from a detailed decision modeling and part load description.
Regulators can investigate the effect of certain laws on rational decisions.
Practitioners can use the framework for determining optimal BES systems and analyzing trade-offs between economic and ecologic objectives.

The rest of this paper is structured as follows.
Section~\ref{sec:Modeling} describes the developed optimization model.
Afterwards, Section~\ref{sec:Case study} presents a case study in which the effects of all implemented German regulations are analyzed in detail.
Finally, the findings are summarized and an outlook for future research is given in Section~\ref{sec:Conclusions}.

The models and application described in this paper can be downloaded for free from \url{https://github.com/RWTH-EBC/BESopt}. (The currently private repository will be made public upon acceptance of this paper.)